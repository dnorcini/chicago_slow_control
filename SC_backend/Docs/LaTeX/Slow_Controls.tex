\documentclass[12pt]{report}
\pagestyle{headings}
\usepackage{graphicx}
\usepackage{longtable}
\usepackage{latexsym}
\usepackage{color}
\usepackage{array}
\usepackage{multirow}
\usepackage{hyperref}
\usepackage{wasysym}
\bibliographystyle{vitae}


%%\setlength{\oddsidemargin}{0in}
%%\setlength{\evensidemargin}{0in}
%%\setlength{\textwidth}{6.5in}
%%\setlength{\topmargin}{-.4in}
%%\setlength{\headheight}{0in}
%%\setlength{\textheight}{9in}


\title{Astro Slow Control}

\author{James Nikkel\thanks{james.nikkel@yale.edu} \\
  Yale University}

\begin{document}

\maketitle

\tableofcontents

\chapter{Introduction}

This document describes the design,  installation, and use of the experimental slow control system.  This systems is designed to monitor and control experimental telemetry for small to medium sized experiments with up to a few thousand sensors.

\section{Architecture}

The control systems centres around a single MySQL database that is populated by an array of independent processes that may or may not run on the same machine where the master database resides. 

The master database is continuously replicated (copied) in real time to a variety of slave databases that exist on physically separate machines, some in remote locations. This provides a real-time backup of the master as well as reduces communication load on the master as general queries about detector conditions can be handled by the slaves rather than the master. 

\begin{figure}[!ht]
\begin{center}
\includegraphics[width=10cm]{"figs/SC_arch"}
\caption{Schematic of the slow control system}
\label{fig:schem}
\end{center}
\end{figure}

Out of bounds alarms (value and rate) are set via a web-based front end. The alarms work by monitoring the database for out of bounds conditions then setting a flag in the database indicating that there is an alarm, and adding an alert message to the log. A second process monitors the log and sends out messages to a pre-determined set of people if a new alert message is found. Alarms are sent via email, SMS, as well as a local auditory and visual siren. If an alarm is not acknowledged within a preset time limit, additional alerts are sent out, eventually adding additional people to the distribution list. 

All processes log themselves with the database and a pair of Watchdogs ensure that all required processes (and each other) are running. The second Watchdog resides on a remote machine to verify that the network is active. The remote Watchdog does not rely on communication with the database to send out alarms. 

The primary user interface uses simple forms based on php and html. This means that the simplest web browser can have maximum control over the detector (including phones) without the need for client side plugins such as JAVA or AJAX. Any number of other front ends can be added with the only requirement that they be able to communicate with the MySQL server. Both rudimentary Matlab and ROOT interfaces have been developed by others.
 

\chapter{Installation}

\section{Hardware}

The slow control system consists of one or more PCs running Linux.  These can be pretty modest;  two cores, at least 2 GB of RAM, a couple of hard drives is not a bad idea.  A machine with integrated video is a good idea as they normally use less power than a discrete video card.  Get a good UPS.  

One machine is designated the `master'.  It contains the master database, and all database writes are done there.  One set of computers (which may contain the master) are used for the hardware interfaces, though each non-master machine should have its own watchdog system.  Another set of machines (which may also contain the master) run the frontend web server(s), and contain a replicated copy (slave) of the master database.  If there are a large number of slaves, it can be a good idea to distribute the load in a tree fashion, with a small number of primary slaves, then secondary slaves replicating them.  Alternatively, for small control systems, everything (frontend and backend) can be done on a single machine.

There are a few consideration to make for the hardware interfaces.  

If you need GPIB, get a National Instruments GPIB card, and install the driver that is included in the "extras" directory.

If you wish to use RS232 to USB converters, purchase a converter box with more ports than you need.  Due to the way USB ports are enumerated, adding boxes in the future can be somewhat painful.

A much better option for serial devices is to use a serial device server.   This is a small computer with one or two serial ports that effectively converts serial to tcp.  They can usually be configured for RS232, RS485 or RS488.

If you want webcams, try to buy one of the supported ones from this list:

http://mxhaard.free.fr/spca5xx.html

If you have ethernet devices, get a second network card, or make sure you purchase a computer with 2 built in.  Do not attach network capable instruments to the internet unless you really know what you are doing.  Many run older versions of Windows that can not be patched and are a security risk.  You'll also need an ethernet switch to connect the devices and computer together.  

\section{Pre-requisite Software}

For operating system, any modern Linux is fine.  OpenSUSE~\footnote{http://software.opensuse.org} is a good all around  distribution with some nice configuration tools.

During installation, you need to install:

\begin{itemize}
\item{MySql}
\item{All the MYSQL client development packages}
\item{Apache  (or any other webserver)}
\item{php}
\item{php-mysql}
\item{phpMyAdmin (not necessary but convenient)}
\item{C/C++ dev tools}
\item{gspca  (for cameras)}
\item{git}
\item{kernel source (for hardware that uses kernel modules)}
\end{itemize}

After installation carefully go through the run levels to only run the services you need (webserver, mysql, ntp).  Also run a systems update to make sure the system is up to date.

Systems are normally run headless and at runlevel 3.  There isn't any point in having X running in the background if it can't or won't ever be called on.

\section{Installing the `Backend'}
\label{sec:inst-backend}

Now for the actual slow control software.  There are two repositories to make the system easier to use and keep updated.  All the documentation (including this one) is kept in the backend repository.  The front end can be checked out directly into your web server directory and contains only php code. To check out the slow control backend, type:

\texttt{git clone https://bitbucket.org/jnikkel/astro-slow-control.git}

\noindent  
go to:

\texttt{SC\_backend/slow\_control\_code/lib}

\noindent  
and run make.  Either add this directory to your library path, or make a link to /texttt{libslow\_control.so} in a directory that is in your library path (\texttt{/usr/lib}, or \texttt{/usr/local/lib} for example).   Then go to: 

\texttt{SC\_backend/slow\_control\_code}

\noindent
 and run make again.  Copy \texttt{slow\_control\_db\_conf.ini} from \texttt{SC\_backend/Docs} to \texttt{/etc} and edit appropriately.  The file looks like this: \\

 \noindent
 \texttt{[client]\\
host=localhost\\
database=control\\
password=control\_user\_password\\
user=control\_user\\
port=3306\\
}
 
 `host' is the location of the master database, which is `localhost' if it resides on the same computer and a resolvable IP address otherwise.  You will want to change the password to something else, but be aware that it is stored as plain text here.  The port number for MySQL is 3306 by default, but you may use some other port on your system.  The MySQL user name for the slow control software is `control\_user' but you can change that to something else here.  Similarly, you can change the name of the database to something other than `control'.  If you wish to use a different \texttt{.ini} file name (for multiple installations on a single machine, for example), then you will have to change the definition of `DEF\_DB\_CONF\_FILE' in \texttt{SC\_backend/slow\_control\_code/include/SC\_db\_interface.h}.

 
 The system is not yet usable, you must first set up the master MySQL database.
 
 
 \section{Setting up MySQL}

First make sure that you have a root password set for your MySQL installation.  Log in as root and create your master database.  As per the example above, I'll assume it is called `control'.  Import the SQL structure and data from \texttt{SC\_backend/Docs/control\_DB\_base.sql} into this database.

See appendix~\ref{apx:struc}  for a list of the tables that need to be defined in the control database.  A copy of this database can be found in \texttt{SC\_backend/Docs/}.


\section{Installing the `Frontend'}


Go to your web server directory (with appropriate permissions):

 \texttt{svn co svn://clean.physics.yale.edu/SC\_web}

\noindent
Set permissions so that the  \texttt{jpgraph\_cache} directory can be written to by whatever user your web server uses for php code.  (a+rw is probably safe here, but only this directory). 






\chapter{Usage}

\section{The `Backend'}

The `Back-end' is made up of individual programs that are each dedicated to a small number of tasks.  These programs are largely controlled through the web `Front-end' (see section~\ref{sec:inst-conf}).  The exception to this is the `Watchdog' program which has to be started either manually or by the operating system upon boot-up.  A script for starting the Watchdog is created by running the install script during installation (section~\ref{sec:inst-backend}).  By default, all the processes run detached and with full root privileges.  This is unfortunately normally necessary to enable communication with hardware that uses many third party kernel modules.  If you find this is not necessary, you may edit how the watchdog is started.


\section{The `Frontend'}

\subsection{Users and Privileges}

All users of the Experimental Controls System have a unique user name and password.  The log-in button is always in the upper right of all web pages.  When a user logs onto the controls page, the connecting IP is captured, and user information is loaded from the database.  One item from the database is a `privilege list'.  This list gives that user a set of privileges that allows him or her to control various aspects of the experiment.  For example having the `T-Control' privilege may mean that that user is allowed to change the set point of a temperature controller.  Privileges may also be used in combination, so that an ion gauge may require both `HV' and `Pressure' privileges, for example.

The second access control is based on IP logging.  Certain controls should only be carried out from specific locations.  For example, a high voltage supply may require that the user have `HV' privileges and have a connecting IP that corresponds to the computer sub-net in the same room.  

If a user does not have the correct access privileges, then not only may they not set that control, it does not even appear on the pages.

If a user does not log-in, then they are assigned guest privileges.  A small subset of pages may be viewed, but no controls are accessible.


\begin{figure}[!ht]
\begin{center}
\includegraphics[width=8cm]{"figs/SC_ex"}
\caption{Screen capture of the `Calcs' page showing a number of the standard
	interface elements.}
\label{fig:sc_ex}
\end{center}
\end{figure}

\subsection{Displays}

There are a number of display pages that are useful for different situations and sensors.  The `Plots' page plots each sensor value over a fixed time interval on its own graph.  The time interval can be edited manually, or a number of short-cuts can be entered into the `Begin time' and `End time' fields.  (Show how -2h works here)

The `MPlot' page plots up to 7 lines on the same graph, each with its own vertical axis.  The `Scatter' page plots one sensor value against another over the given time interval.  This can be useful for looking at pressure vs. volume or other phase information.

The `Text' page is a simple table of the current values of the selected sensors.  At the bottom of the page one can choose a number to increase the averaging that is done.  This can be useful if you want to monitor a sensor that is noisy but read out quickly.
  
\subsection{Controls}


\begin{figure}[!ht]
\begin{center}
\includegraphics[width=13.5cm]{"figs/SC_control"}
\caption{Screen capture of the `Control' page showing the three types of controls.}
\label{fig:sc_control}
\end{center}
\end{figure}


If a sensor is designated as `settable` in its configuration, then it will appear on this page.  Settable sensors can be configured in one of two ways.  If the  set-point is a continuous variable, like a temperature or flow, then it will appear on the controls page as a text field, and only accept a float as an input. In this case, the units are set in the `units` field on the configuration page.

If the control is discrete, then either buttons will appear on the page if there are  fewer than five choices, or a pull down menu will be available if there are  five or more choices.  The configuration of the controls is described in section~\ref{sec:sens-conf}.  

Buttons are controlled by clicking them once, pull downs are set by selecting the  desired setting and clicking the `Change' button.  Numeric values are entered by typing, and hitting enter when complete.


\subsection{HV Crate Controls}

\begin{figure}[!ht]
\begin{center}
\includegraphics[width=13.5cm]{"figs/Crate_HV_1"}\\
$\approx$\\
\includegraphics[width=13.5cm]{"figs/Crate_HV_2"}
\caption{Screen capture of the `HV Crate' page showing the display and control 
	elements.}
\label{fig:sc_crate}
\end{center}
\end{figure}

If a high voltage crate is to be used as part of the slow control, there is a specialized page that may be turned on by setting `int1' of the `have\_crate`' entry in the `globals'  table to 1.  

Figure~\ref{fig:sc_crate} is a screen capture of the control page with the centre  section cut out to save some space.  At the top of the page below the header is the master On/Off control.  The rest of the page is arrange the same way the cards are in the crate.  The cell at the top of each column contains a single control for toggling that card on and off.  Red indicates that it is off, green indicates that it is on.  There is  also a short designator for the slot, `Veto�, or `Det� (for Detector) in this case (this text is taken from the `Subtype' in the Sensor Configuration).  `SSt:X� indicates the status of that card.  `Sts:0� meansit is okay, non-zero values indicate some sort of problem or activity (such as ramping up a voltage). 

Below the slot level cell are the individual channel control cells.  Each channel can be toggled on or off by clicking the red (off$\rightarrow$on) or green (on$\rightarrow$off) button.   If a channel has tripped off, the cell will turn to red and a reset button will appear.  An alarm will be sent out, and the channel will be automatically set to Off.  To reset the channel, fix the problem, and click the `Reset� button.  The voltage will not come back on until you click the red (off$\rightarrow$on) button on that channel.

Below that are the voltage and current outputs.  Below that is the voltage set-point.  Enter a new value and hit enter to commit it.  If you change multiple fields then hit enter, only the last one will be changed.

At the bottom of each column are fields for the Current Trip value, and the Ramp Rate.  These values are set for the entire slot (all 12 channels)

If a slot is not occupied or is otherwise incapacitated, then there will be a blank column.  In this example, Slot 3 is empty.  A slot can be disabled by clicking the `Hide� button of the sensor in the Sensor Configuration corresponding to that slot.

\subsection{Alarms}

Alarms are an crucial part of the monitoring and control system.  Each sensor entry may have up to 4 alarms set for it.  A high and low value alarm, and a high and low rate alarm.  The rate alarm becomes available if the ``Show Rate'' checkbox is selected in the sensor configuration for that entry (see Section~\ref{sec:sens-conf}).  

\begin{figure}[!ht]
\begin{center}
\includegraphics[width=13.5cm]{"figs/SC_alarms"}
\caption{Screen capture of the `Alarms' page showing the three sensors and their alarm settings.}
\label{fig:sc_alarm}
\end{center}
\end{figure}

Figure~\ref{fig:sc_alarm} shows the alarms page along with three different sensors and their alarm settings.  The current value and the last update time are shown along with the alarm configuration. The sensor on the left has a high alarm set, so that if the value (2.47V as of 10 seconds ago) goes above 12, an alarm is sent.  The middle entry has a low alarm set point, as well as a high rate set point.  The sensor on the right has no alarms set.  

There are two primary components to the alarm system.  The first is the ``Alarm Trip Daemon" that monitors the sensor values and alarm set points and makes an entry into the systems log if a sensor goes out of bounds.  The second is the ``Alarm Alert Daemon'' that monitors the systems log and sends out email and text messages to the ``on-call'' users to alert them to the problem.  This daemon also sets a flag that there is an alarm set.  This creates an ``Acknowledge Current Alarm'' button on the alarms page and also forces each page to play a siren sound.  To stop the alarm, any regular user (i.e. not a guest account user) can click the ``Acknowledge Current Alarm'' button to reset the alarm flag and an entry with that username goes to the systems log.  It is a good idea to fix the problem that caused the alarm to go off first to prevent it from sounding repeatedly.  

\begin{figure}[!ht]
\begin{center}
\includegraphics[width=13.5cm]{"figs/SC_alarm_button"}
\caption{Screen capture of the `Alarms' page with one sensor below the alarm set point.  The black bar on the upper left is the media player that plays an alarm sound and below that is the alarm acknowledge button.}
\label{fig:sc_alarm_button}
\end{center}
\end{figure}

Figure~\ref{fig:sc_alarm_button} shows the alarms page along with 

\subsection{Configurations}

\subsubsection{Sensor Configuration}
\label{sec:sens-conf}

Here one can edit the settings that are associated with each individual sensor. All changes are saved to the systems log along with the user who made the change.  

\subsubsection{Instrument Configuration}
\label{sec:inst-conf}

Here one can edit and control the `Instruments' that make up the control system. Each is a stand alone software program that may or may not communicate with one or more physical instrument.  For example there may be one entry for a temperature controller that has four thermometers and a heater, one program may communicate with six different ModBUS devices, and another may just perform some purely software task.  The alarm system programs as well as the Watchdogs are all controlled and configured on this page.

\appendix

\chapter{Safety}
A word should be said about personnel safety with regards to the Experimental Controls System.  Since this system is open to a network and is necessarily very complicated, one should assume that it is possible for the system to become compromised, and that bugs are certain to be present.  It can not be assumed that hardware attached to the experimental controls system can be made safe through its actions or in spite of them.  

Computer systems that are used for life safety, in medical or flight applications, for example, go through rigorous testing both at the hardware and software level.  No such rigour is possible for this controls system.  To ensure safety of personnel, external fail-safe systems and interlocks are required when controlling hardware that may cause injury.

For example, this controls system may control a high voltage power supply. If the conductor connected to the supply is being installed or undergoing maintenance, then that supply must be powered off (from the mains, for example), or assumed to be live.  It is not sufficient to assume that the control system has turned the voltage down to a safe level or that it will not turn it up to an unsafe level spontaneously.  

Gas control systems are another case where unsafe situations may arise.  While it may be convenient to have remote control operation of valves and flow rates available, that is not a substitute for reliable pressure relief valves and rupture devices.  

The Experimental Controls System is a powerful tool for control and monitoring but it does not replace or mitigate any regular safety procedures or devices.

\chapter{Control Database Structure}
\label{apx:struc}
The following latex tables show the structure (but not data) of the control database.  A full SQL copy, including some necessary data entries, can be found as `control\_DB\_base.sql'  in the Docs directory.

% phpMyAdmin LaTeX Dump
% version 4.4.15.8
% https://www.phpmyadmin.net
%
% Host: localhost
% Generation Time: Sep 22, 2016 at 01:51 PM
% Server version: 10.0.25-MariaDB
% PHP Version: 5.5.14
% 
% Database: 'ASC_control'
% 

%
% Structure: daq_control
%
 \begin{longtable}{|l|c|c|c|l|} 
 \caption{Structure of table daq\_control} \label{tab:daq_control-structure} \\
 \hline \multicolumn{1}{|c|}{\textbf{Column}} & \multicolumn{1}{|c|}{\textbf{Type}} & \multicolumn{1}{|c|}{\textbf{Null}} & \multicolumn{1}{|c|}{\textbf{Default}} & \multicolumn{1}{|c|}{\textbf{Comments}} \\ \hline \hline
\endfirsthead
 \caption{Structure of table daq\_control (continued)} \\ 
 \hline \multicolumn{1}{|c|}{\textbf{Column}} & \multicolumn{1}{|c|}{\textbf{Type}} & \multicolumn{1}{|c|}{\textbf{Null}} & \multicolumn{1}{|c|}{\textbf{Default}} & \multicolumn{1}{|c|}{\textbf{Comments}} \\ \hline \hline \endhead \endfoot 
\textbf{\textit{entry\_id}} & int(11) & No &  \\ \hline 
lug\_entry\_id & int(11) & No &  \\ \hline 
utc & int(11) & No &  \\ \hline 
acquire\_now & tinyint(1) & No &  \\ \hline 
 \end{longtable}

%
% Data: daq_control
%
 \begin{longtable}{|l|l|l|l|} 
 \hline \endhead \hline \endfoot \hline 
 \caption{Content of table daq\_control} \label{tab:daq_control-data} \\\hline \multicolumn{1}{|c|}{\textbf{entry\_id}} & \multicolumn{1}{|c|}{\textbf{lug\_entry\_id}} & \multicolumn{1}{|c|}{\textbf{utc}} & \multicolumn{1}{|c|}{\textbf{acquire\_now}} \\ \hline \hline  \endfirsthead 
\caption{Content of table daq\_control (continued)} \\ \hline \multicolumn{1}{|c|}{\textbf{entry\_id}} & \multicolumn{1}{|c|}{\textbf{lug\_entry\_id}} & \multicolumn{1}{|c|}{\textbf{utc}} & \multicolumn{1}{|c|}{\textbf{acquire\_now}} \\ \hline \hline \endhead \endfoot
 \end{longtable}

%
% Structure: daq_control_tmp
%
 \begin{longtable}{|l|c|c|c|l|} 
 \caption{Structure of table daq\_control\_tmp} \label{tab:daq_control_tmp-structure} \\
 \hline \multicolumn{1}{|c|}{\textbf{Column}} & \multicolumn{1}{|c|}{\textbf{Type}} & \multicolumn{1}{|c|}{\textbf{Null}} & \multicolumn{1}{|c|}{\textbf{Default}} & \multicolumn{1}{|c|}{\textbf{Comments}} \\ \hline \hline
\endfirsthead
 \caption{Structure of table daq\_control\_tmp (continued)} \\ 
 \hline \multicolumn{1}{|c|}{\textbf{Column}} & \multicolumn{1}{|c|}{\textbf{Type}} & \multicolumn{1}{|c|}{\textbf{Null}} & \multicolumn{1}{|c|}{\textbf{Default}} & \multicolumn{1}{|c|}{\textbf{Comments}} \\ \hline \hline \endhead \endfoot 
utc & int(11) & No &  \\ \hline 
user\_name & text & No &  \\ \hline 
tag & text & No &  \\ \hline 
acquire\_now & tinyint(1) & No &  \\ \hline 
xml\_settings\_file & text & No &  \\ \hline 
end\_status & text & No &  \\ \hline 
 \end{longtable}

%
% Data: daq_control_tmp
%
 \begin{longtable}{|l|l|l|l|l|l|} 
 \hline \endhead \hline \endfoot \hline 
 \caption{Content of table daq\_control\_tmp} \label{tab:daq_control_tmp-data} \\\hline \multicolumn{1}{|c|}{\textbf{utc}} & \multicolumn{1}{|c|}{\textbf{user\_name}} & \multicolumn{1}{|c|}{\textbf{tag}} & \multicolumn{1}{|c|}{\textbf{acquire\_now}} & \multicolumn{1}{|c|}{\textbf{xml\_settings\_file}} & \multicolumn{1}{|c|}{\textbf{end\_status}} \\ \hline \hline  \endfirsthead 
\caption{Content of table daq\_control\_tmp (continued)} \\ \hline \multicolumn{1}{|c|}{\textbf{utc}} & \multicolumn{1}{|c|}{\textbf{user\_name}} & \multicolumn{1}{|c|}{\textbf{tag}} & \multicolumn{1}{|c|}{\textbf{acquire\_now}} & \multicolumn{1}{|c|}{\textbf{xml\_settings\_file}} & \multicolumn{1}{|c|}{\textbf{end\_status}} \\ \hline \hline \endhead \endfoot
 \end{longtable}

%
% Structure: daq_presets
%
 \begin{longtable}{|l|c|c|c|l|} 
 \caption{Structure of table daq\_presets} \label{tab:daq_presets-structure} \\
 \hline \multicolumn{1}{|c|}{\textbf{Column}} & \multicolumn{1}{|c|}{\textbf{Type}} & \multicolumn{1}{|c|}{\textbf{Null}} & \multicolumn{1}{|c|}{\textbf{Default}} & \multicolumn{1}{|c|}{\textbf{Comments}} \\ \hline \hline
\endfirsthead
 \caption{Structure of table daq\_presets (continued)} \\ 
 \hline \multicolumn{1}{|c|}{\textbf{Column}} & \multicolumn{1}{|c|}{\textbf{Type}} & \multicolumn{1}{|c|}{\textbf{Null}} & \multicolumn{1}{|c|}{\textbf{Default}} & \multicolumn{1}{|c|}{\textbf{Comments}} \\ \hline \hline \endhead \endfoot 
\textbf{\textit{entry\_id}} & int(11) & No &  \\ \hline 
preset & mediumblob & No &  \\ \hline 
 \end{longtable}

%
% Data: daq_presets
%
 \begin{longtable}{|l|l|} 
 \hline \endhead \hline \endfoot \hline 
 \caption{Content of table daq\_presets} \label{tab:daq_presets-data} \\\hline \multicolumn{1}{|c|}{\textbf{entry\_id}} & \multicolumn{1}{|c|}{\textbf{preset}} \\ \hline \hline  \endfirsthead 
\caption{Content of table daq\_presets (continued)} \\ \hline \multicolumn{1}{|c|}{\textbf{entry\_id}} & \multicolumn{1}{|c|}{\textbf{preset}} \\ \hline \hline \endhead \endfoot
 \end{longtable}

%
% Structure: dqm_channel_data
%
 \begin{longtable}{|l|c|c|c|l|} 
 \caption{Structure of table dqm\_channel\_data} \label{tab:dqm_channel_data-structure} \\
 \hline \multicolumn{1}{|c|}{\textbf{Column}} & \multicolumn{1}{|c|}{\textbf{Type}} & \multicolumn{1}{|c|}{\textbf{Null}} & \multicolumn{1}{|c|}{\textbf{Default}} & \multicolumn{1}{|c|}{\textbf{Comments}} \\ \hline \hline
\endfirsthead
 \caption{Structure of table dqm\_channel\_data (continued)} \\ 
 \hline \multicolumn{1}{|c|}{\textbf{Column}} & \multicolumn{1}{|c|}{\textbf{Type}} & \multicolumn{1}{|c|}{\textbf{Null}} & \multicolumn{1}{|c|}{\textbf{Default}} & \multicolumn{1}{|c|}{\textbf{Comments}} \\ \hline \hline \endhead \endfoot 
utc & int(11) & No &  \\ \hline 
runs & text & No &  \\ \hline 
prefixes & text & No &  \\ \hline 
dat\_files & text & No &  \\ \hline 
avg\_event\_rate & float & No &  \\ \hline 
avg\_livetime & float & No &  \\ \hline 
pulse\_rates & text & No &  \\ \hline 
avg\_pulse\_lengths & text & No &  \\ \hline 
avg\_baselines\_mv & text & No &  \\ \hline 
 \end{longtable}

%
% Data: dqm_channel_data
%
 \begin{longtable}{|l|l|l|l|l|l|l|l|l|} 
 \hline \endhead \hline \endfoot \hline 
 \caption{Content of table dqm\_channel\_data} \label{tab:dqm_channel_data-data} \\\hline \multicolumn{1}{|c|}{\textbf{utc}} & \multicolumn{1}{|c|}{\textbf{runs}} & \multicolumn{1}{|c|}{\textbf{prefixes}} & \multicolumn{1}{|c|}{\textbf{dat\_files}} & \multicolumn{1}{|c|}{\textbf{avg\_event\_rate}} & \multicolumn{1}{|c|}{\textbf{avg\_livetime}} & \multicolumn{1}{|c|}{\textbf{pulse\_rates}} & \multicolumn{1}{|c|}{\textbf{avg\_pulse\_lengths}} & \multicolumn{1}{|c|}{\textbf{avg\_baselines\_mv}} \\ \hline \hline  \endfirsthead 
\caption{Content of table dqm\_channel\_data (continued)} \\ \hline \multicolumn{1}{|c|}{\textbf{utc}} & \multicolumn{1}{|c|}{\textbf{runs}} & \multicolumn{1}{|c|}{\textbf{prefixes}} & \multicolumn{1}{|c|}{\textbf{dat\_files}} & \multicolumn{1}{|c|}{\textbf{avg\_event\_rate}} & \multicolumn{1}{|c|}{\textbf{avg\_livetime}} & \multicolumn{1}{|c|}{\textbf{pulse\_rates}} & \multicolumn{1}{|c|}{\textbf{avg\_pulse\_lengths}} & \multicolumn{1}{|c|}{\textbf{avg\_baselines\_mv}} \\ \hline \hline \endhead \endfoot
 \end{longtable}

%
% Structure: dqm_control
%
 \begin{longtable}{|l|c|c|c|l|} 
 \caption{Structure of table dqm\_control} \label{tab:dqm_control-structure} \\
 \hline \multicolumn{1}{|c|}{\textbf{Column}} & \multicolumn{1}{|c|}{\textbf{Type}} & \multicolumn{1}{|c|}{\textbf{Null}} & \multicolumn{1}{|c|}{\textbf{Default}} & \multicolumn{1}{|c|}{\textbf{Comments}} \\ \hline \hline
\endfirsthead
 \caption{Structure of table dqm\_control (continued)} \\ 
 \hline \multicolumn{1}{|c|}{\textbf{Column}} & \multicolumn{1}{|c|}{\textbf{Type}} & \multicolumn{1}{|c|}{\textbf{Null}} & \multicolumn{1}{|c|}{\textbf{Default}} & \multicolumn{1}{|c|}{\textbf{Comments}} \\ \hline \hline \endhead \endfoot 
utc\_time & int(11) & No &  \\ \hline 
 \end{longtable}

%
% Data: dqm_control
%
 \begin{longtable}{|l|} 
 \hline \endhead \hline \endfoot \hline 
 \caption{Content of table dqm\_control} \label{tab:dqm_control-data} \\\hline \multicolumn{1}{|c|}{\textbf{utc\_time}} \\ \hline \hline  \endfirsthead 
\caption{Content of table dqm\_control (continued)} \\ \hline \multicolumn{1}{|c|}{\textbf{utc\_time}} \\ \hline \hline \endhead \endfoot
 \end{longtable}

%
% Structure: dqm_event_data
%
 \begin{longtable}{|l|c|c|c|l|} 
 \caption{Structure of table dqm\_event\_data} \label{tab:dqm_event_data-structure} \\
 \hline \multicolumn{1}{|c|}{\textbf{Column}} & \multicolumn{1}{|c|}{\textbf{Type}} & \multicolumn{1}{|c|}{\textbf{Null}} & \multicolumn{1}{|c|}{\textbf{Default}} & \multicolumn{1}{|c|}{\textbf{Comments}} \\ \hline \hline
\endfirsthead
 \caption{Structure of table dqm\_event\_data (continued)} \\ 
 \hline \multicolumn{1}{|c|}{\textbf{Column}} & \multicolumn{1}{|c|}{\textbf{Type}} & \multicolumn{1}{|c|}{\textbf{Null}} & \multicolumn{1}{|c|}{\textbf{Default}} & \multicolumn{1}{|c|}{\textbf{Comments}} \\ \hline \hline \endhead \endfoot 
utc\_time & int(11) & No &  \\ \hline 
run & int(11) & No &  \\ \hline 
prefix & text & No &  \\ \hline 
event\_rate & float & No &  \\ \hline 
livetime & float & No &  \\ \hline 
deadtime & float & No &  \\ \hline 
 \end{longtable}

%
% Data: dqm_event_data
%
 \begin{longtable}{|l|l|l|l|l|l|} 
 \hline \endhead \hline \endfoot \hline 
 \caption{Content of table dqm\_event\_data} \label{tab:dqm_event_data-data} \\\hline \multicolumn{1}{|c|}{\textbf{utc\_time}} & \multicolumn{1}{|c|}{\textbf{run}} & \multicolumn{1}{|c|}{\textbf{prefix}} & \multicolumn{1}{|c|}{\textbf{event\_rate}} & \multicolumn{1}{|c|}{\textbf{livetime}} & \multicolumn{1}{|c|}{\textbf{deadtime}} \\ \hline \hline  \endfirsthead 
\caption{Content of table dqm\_event\_data (continued)} \\ \hline \multicolumn{1}{|c|}{\textbf{utc\_time}} & \multicolumn{1}{|c|}{\textbf{run}} & \multicolumn{1}{|c|}{\textbf{prefix}} & \multicolumn{1}{|c|}{\textbf{event\_rate}} & \multicolumn{1}{|c|}{\textbf{livetime}} & \multicolumn{1}{|c|}{\textbf{deadtime}} \\ \hline \hline \endhead \endfoot
 \end{longtable}

%
% Structure: dqm_event_display
%
 \begin{longtable}{|l|c|c|c|l|} 
 \caption{Structure of table dqm\_event\_display} \label{tab:dqm_event_display-structure} \\
 \hline \multicolumn{1}{|c|}{\textbf{Column}} & \multicolumn{1}{|c|}{\textbf{Type}} & \multicolumn{1}{|c|}{\textbf{Null}} & \multicolumn{1}{|c|}{\textbf{Default}} & \multicolumn{1}{|c|}{\textbf{Comments}} \\ \hline \hline
\endfirsthead
 \caption{Structure of table dqm\_event\_display (continued)} \\ 
 \hline \multicolumn{1}{|c|}{\textbf{Column}} & \multicolumn{1}{|c|}{\textbf{Type}} & \multicolumn{1}{|c|}{\textbf{Null}} & \multicolumn{1}{|c|}{\textbf{Default}} & \multicolumn{1}{|c|}{\textbf{Comments}} \\ \hline \hline \endhead \endfoot 
utc & int(11) & No &  \\ \hline 
picture & longblob & No &  \\ \hline 
 \end{longtable}

%
% Data: dqm_event_display
%
 \begin{longtable}{|l|l|} 
 \hline \endhead \hline \endfoot \hline 
 \caption{Content of table dqm\_event\_display} \label{tab:dqm_event_display-data} \\\hline \multicolumn{1}{|c|}{\textbf{utc}} & \multicolumn{1}{|c|}{\textbf{picture}} \\ \hline \hline  \endfirsthead 
\caption{Content of table dqm\_event\_display (continued)} \\ \hline \multicolumn{1}{|c|}{\textbf{utc}} & \multicolumn{1}{|c|}{\textbf{picture}} \\ \hline \hline \endhead \endfoot
 \end{longtable}

%
% Structure: globals
%
 \begin{longtable}{|l|c|c|c|l|} 
 \caption{Structure of table globals} \label{tab:globals-structure} \\
 \hline \multicolumn{1}{|c|}{\textbf{Column}} & \multicolumn{1}{|c|}{\textbf{Type}} & \multicolumn{1}{|c|}{\textbf{Null}} & \multicolumn{1}{|c|}{\textbf{Default}} & \multicolumn{1}{|c|}{\textbf{Comments}} \\ \hline \hline
\endfirsthead
 \caption{Structure of table globals (continued)} \\ 
 \hline \multicolumn{1}{|c|}{\textbf{Column}} & \multicolumn{1}{|c|}{\textbf{Type}} & \multicolumn{1}{|c|}{\textbf{Null}} & \multicolumn{1}{|c|}{\textbf{Default}} & \multicolumn{1}{|c|}{\textbf{Comments}} \\ \hline \hline \endhead \endfoot 
\textbf{\textit{name}} & varchar(16) & No &  \\ \hline 
int1 & int(11) & No & 0 \\ \hline 
int2 & int(11) & No & 0 \\ \hline 
int3 & int(11) & No & 0 \\ \hline 
int4 & int(11) & No & 0 \\ \hline 
double1 & double & No & 0 \\ \hline 
double2 & double & No & 0 \\ \hline 
double3 & double & No & 0 \\ \hline 
double4 & double & No & 0 \\ \hline 
string1 & varchar(16) & Yes & NULL \\ \hline 
string2 & varchar(16) & Yes & NULL \\ \hline 
string3 & varchar(16) & Yes & NULL \\ \hline 
string4 & varchar(16) & Yes & NULL \\ \hline 
 \end{longtable}

%
% Data: globals
%
 \begin{longtable}{|l|l|l|l|l|l|l|l|l|l|l|l|l|} 
 \hline \endhead \hline \endfoot \hline 
 \caption{Content of table globals} \label{tab:globals-data} \\\hline \multicolumn{1}{|c|}{\textbf{name}} & \multicolumn{1}{|c|}{\textbf{int1}} & \multicolumn{1}{|c|}{\textbf{int2}} & \multicolumn{1}{|c|}{\textbf{int3}} & \multicolumn{1}{|c|}{\textbf{int4}} & \multicolumn{1}{|c|}{\textbf{double1}} & \multicolumn{1}{|c|}{\textbf{double2}} & \multicolumn{1}{|c|}{\textbf{double3}} & \multicolumn{1}{|c|}{\textbf{double4}} & \multicolumn{1}{|c|}{\textbf{string1}} & \multicolumn{1}{|c|}{\textbf{string2}} & \multicolumn{1}{|c|}{\textbf{string3}} & \multicolumn{1}{|c|}{\textbf{string4}} \\ \hline \hline  \endfirsthead 
\caption{Content of table globals (continued)} \\ \hline \multicolumn{1}{|c|}{\textbf{name}} & \multicolumn{1}{|c|}{\textbf{int1}} & \multicolumn{1}{|c|}{\textbf{int2}} & \multicolumn{1}{|c|}{\textbf{int3}} & \multicolumn{1}{|c|}{\textbf{int4}} & \multicolumn{1}{|c|}{\textbf{double1}} & \multicolumn{1}{|c|}{\textbf{double2}} & \multicolumn{1}{|c|}{\textbf{double3}} & \multicolumn{1}{|c|}{\textbf{double4}} & \multicolumn{1}{|c|}{\textbf{string1}} & \multicolumn{1}{|c|}{\textbf{string2}} & \multicolumn{1}{|c|}{\textbf{string3}} & \multicolumn{1}{|c|}{\textbf{string4}} \\ \hline \hline \endhead \endfoot
Master\_alarm & 0 & 0 & 0 & 0 & 0 & 0 & 0 & 0 &  &  &  &  \\ \hline 
Title & 0 & 0 & 0 & 0 & 0 & 0 & 0 & 0 & Slow Control Sys & \textit{NULL} & \textit{NULL} & \textit{NULL} \\ \hline 
have\_TS & 0 & 0 & 0 & 0 & 0 & 0 & 0 & 0 & \textit{NULL} & \textit{NULL} & \textit{NULL} & \textit{NULL} \\ \hline 
have\_RGA & 0 & 0 & 0 & 0 & 0 & 0 & 0 & 0 & \textit{NULL} & \textit{NULL} & \textit{NULL} & \textit{NULL} \\ \hline 
have\_Cams & 1 & 0 & 0 & 0 & 0 & 0 & 0 & 0 & \textit{NULL} & \textit{NULL} & \textit{NULL} & \textit{NULL} \\ \hline 
have\_LB & 1 & 0 & 0 & 0 & 0 & 0 & 0 & 0 & \textit{NULL} & \textit{NULL} & \textit{NULL} & \textit{NULL} \\ \hline 
web\_bg\_colour & 0 & 0 & 0 & 0 & 0 & 0 & 0 & 0 & wheat & \textit{NULL} & \textit{NULL} & \textit{NULL} \\ \hline 
web\_text\_colour & 0 & 0 & 0 & 0 & 0 & 0 & 0 & 0 & navy & \textit{NULL} & \textit{NULL} & \textit{NULL} \\ \hline 
have\_HV\_crate & 0 & 0 & 0 & 0 & 0 & 0 & 0 & 0 & \textit{NULL} & \textit{NULL} & \textit{NULL} & \textit{NULL} \\ \hline 
 \end{longtable}

%
% Structure: lug_categories
%
 \begin{longtable}{|l|c|c|c|l|} 
 \caption{Structure of table lug\_categories} \label{tab:lug_categories-structure} \\
 \hline \multicolumn{1}{|c|}{\textbf{Column}} & \multicolumn{1}{|c|}{\textbf{Type}} & \multicolumn{1}{|c|}{\textbf{Null}} & \multicolumn{1}{|c|}{\textbf{Default}} & \multicolumn{1}{|c|}{\textbf{Comments}} \\ \hline \hline
\endfirsthead
 \caption{Structure of table lug\_categories (continued)} \\ 
 \hline \multicolumn{1}{|c|}{\textbf{Column}} & \multicolumn{1}{|c|}{\textbf{Type}} & \multicolumn{1}{|c|}{\textbf{Null}} & \multicolumn{1}{|c|}{\textbf{Default}} & \multicolumn{1}{|c|}{\textbf{Comments}} \\ \hline \hline \endhead \endfoot 
\textbf{\textit{l\_c\_indx}} & int(11) & No &  \\ \hline 
category & varchar(40) & No &  \\ \hline 
subcategory & varchar(40) & No & General \\ \hline 
 \end{longtable}

%
% Data: lug_categories
%
 \begin{longtable}{|l|l|l|} 
 \hline \endhead \hline \endfoot \hline 
 \caption{Content of table lug\_categories} \label{tab:lug_categories-data} \\\hline \multicolumn{1}{|c|}{\textbf{l\_c\_indx}} & \multicolumn{1}{|c|}{\textbf{category}} & \multicolumn{1}{|c|}{\textbf{subcategory}} \\ \hline \hline  \endfirsthead 
\caption{Content of table lug\_categories (continued)} \\ \hline \multicolumn{1}{|c|}{\textbf{l\_c\_indx}} & \multicolumn{1}{|c|}{\textbf{category}} & \multicolumn{1}{|c|}{\textbf{subcategory}} \\ \hline \hline \endhead \endfoot
16 & DAQ & General \\ \hline 
15 & Disk & General \\ \hline 
3 & System & General \\ \hline 
4 & System & TempControl \\ \hline 
17 & DAQ & Calibration \\ \hline 
8 & Calibrations & General \\ \hline 
10 & Calibrations & Co60 \\ \hline 
18 & DAQ & Physics \\ \hline 
12 & Calibrations & Co57 \\ \hline 
13 & Calibrations & Ba133 \\ \hline 
 \end{longtable}

%
% Structure: lug_entries
%
 \begin{longtable}{|l|c|c|c|l|} 
 \caption{Structure of table lug\_entries} \label{tab:lug_entries-structure} \\
 \hline \multicolumn{1}{|c|}{\textbf{Column}} & \multicolumn{1}{|c|}{\textbf{Type}} & \multicolumn{1}{|c|}{\textbf{Null}} & \multicolumn{1}{|c|}{\textbf{Default}} & \multicolumn{1}{|c|}{\textbf{Comments}} \\ \hline \hline
\endfirsthead
 \caption{Structure of table lug\_entries (continued)} \\ 
 \hline \multicolumn{1}{|c|}{\textbf{Column}} & \multicolumn{1}{|c|}{\textbf{Type}} & \multicolumn{1}{|c|}{\textbf{Null}} & \multicolumn{1}{|c|}{\textbf{Default}} & \multicolumn{1}{|c|}{\textbf{Comments}} \\ \hline \hline \endhead \endfoot 
\textbf{\textit{entry\_id}} & int(11) & No &  \\ \hline 
important\_flag & tinyint(1) & Yes & 0 \\ \hline 
action\_user & varchar(40) & No &  \\ \hline 
action\_time & int(11) & No &  \\ \hline 
edit\_user & varchar(40) & Yes & NULL \\ \hline 
edit\_time & int(11) & Yes & NULL \\ \hline 
run\_number & int(11) & No &  \\ \hline 
category & varchar(40) & No &  \\ \hline 
subcategory & varchar(40) & No &  \\ \hline 
entry\_description & longtext & No &  \\ \hline 
entry\_image & mediumblob & Yes & NULL \\ \hline 
entry\_image\_thumb & blob & Yes & NULL \\ \hline 
entry\_file & mediumblob & Yes & NULL \\ \hline 
filename & char(40) & Yes & NULL \\ \hline 
source & varchar(40) & Yes & NULL \\ \hline 
strikeme & tinyint(1) & Yes & 0 \\ \hline 
 \end{longtable}

%
% Data: lug_entries
%
 \begin{longtable}{|l|l|l|l|l|l|l|l|l|l|l|l|l|l|l|l|} 
 \hline \endhead \hline \endfoot \hline 
 \caption{Content of table lug\_entries} \label{tab:lug_entries-data} \\\hline \multicolumn{1}{|c|}{\textbf{entry\_id}} & \multicolumn{1}{|c|}{\textbf{important\_flag}} & \multicolumn{1}{|c|}{\textbf{action\_user}} & \multicolumn{1}{|c|}{\textbf{action\_time}} & \multicolumn{1}{|c|}{\textbf{edit\_user}} & \multicolumn{1}{|c|}{\textbf{edit\_time}} & \multicolumn{1}{|c|}{\textbf{run\_number}} & \multicolumn{1}{|c|}{\textbf{category}} & \multicolumn{1}{|c|}{\textbf{subcategory}} & \multicolumn{1}{|c|}{\textbf{entry\_description}} & \multicolumn{1}{|c|}{\textbf{entry\_image}} & \multicolumn{1}{|c|}{\textbf{entry\_image\_thumb}} & \multicolumn{1}{|c|}{\textbf{entry\_file}} & \multicolumn{1}{|c|}{\textbf{filename}} & \multicolumn{1}{|c|}{\textbf{source}} & \multicolumn{1}{|c|}{\textbf{strikeme}} \\ \hline \hline  \endfirsthead 
\caption{Content of table lug\_entries (continued)} \\ \hline \multicolumn{1}{|c|}{\textbf{entry\_id}} & \multicolumn{1}{|c|}{\textbf{important\_flag}} & \multicolumn{1}{|c|}{\textbf{action\_user}} & \multicolumn{1}{|c|}{\textbf{action\_time}} & \multicolumn{1}{|c|}{\textbf{edit\_user}} & \multicolumn{1}{|c|}{\textbf{edit\_time}} & \multicolumn{1}{|c|}{\textbf{run\_number}} & \multicolumn{1}{|c|}{\textbf{category}} & \multicolumn{1}{|c|}{\textbf{subcategory}} & \multicolumn{1}{|c|}{\textbf{entry\_description}} & \multicolumn{1}{|c|}{\textbf{entry\_image}} & \multicolumn{1}{|c|}{\textbf{entry\_image\_thumb}} & \multicolumn{1}{|c|}{\textbf{entry\_file}} & \multicolumn{1}{|c|}{\textbf{filename}} & \multicolumn{1}{|c|}{\textbf{source}} & \multicolumn{1}{|c|}{\textbf{strikeme}} \\ \hline \hline \endhead \endfoot
12 & 0 & james & 1461523662 & james & 1461523690 & 1 & System & General & Test of the log book system. & \textit{NULL} & \textit{NULL} & \textit{NULL} & \textit{NULL} & \textit{NULL} & 0 \\ \hline 
13 & 0 & guest & 1461523701 & james & 1461523729 & 2 & Disk & General & This is also a thing that works. & \textit{NULL} & \textit{NULL} & \textit{NULL} & \textit{NULL} & \textit{NULL} & 0 \\ \hline 
 \end{longtable}

%
% Structure: msg_log
%
 \begin{longtable}{|l|c|c|c|l|} 
 \caption{Structure of table msg\_log} \label{tab:msg_log-structure} \\
 \hline \multicolumn{1}{|c|}{\textbf{Column}} & \multicolumn{1}{|c|}{\textbf{Type}} & \multicolumn{1}{|c|}{\textbf{Null}} & \multicolumn{1}{|c|}{\textbf{Default}} & \multicolumn{1}{|c|}{\textbf{Comments}} \\ \hline \hline
\endfirsthead
 \caption{Structure of table msg\_log (continued)} \\ 
 \hline \multicolumn{1}{|c|}{\textbf{Column}} & \multicolumn{1}{|c|}{\textbf{Type}} & \multicolumn{1}{|c|}{\textbf{Null}} & \multicolumn{1}{|c|}{\textbf{Default}} & \multicolumn{1}{|c|}{\textbf{Comments}} \\ \hline \hline \endhead \endfoot 
\textbf{\textit{msg\_id}} & int(11) & No &  \\ \hline 
time & int(11) & No &  \\ \hline 
ip\_address & varchar(16) & Yes & NULL \\ \hline 
subsys & varchar(16) & Yes & NULL \\ \hline 
msgs & text & Yes & NULL \\ \hline 
type & varchar(16) & Yes & NULL \\ \hline 
is\_error & tinyint(4) & No & 0 \\ \hline 
 \end{longtable}

%
% Data: msg_log
%
 \begin{longtable}{|l|l|l|l|l|l|l|} 
 \hline \endhead \hline \endfoot \hline 
 \caption{Content of table msg\_log} \label{tab:msg_log-data} \\\hline \multicolumn{1}{|c|}{\textbf{msg\_id}} & \multicolumn{1}{|c|}{\textbf{time}} & \multicolumn{1}{|c|}{\textbf{ip\_address}} & \multicolumn{1}{|c|}{\textbf{subsys}} & \multicolumn{1}{|c|}{\textbf{msgs}} & \multicolumn{1}{|c|}{\textbf{type}} & \multicolumn{1}{|c|}{\textbf{is\_error}} \\ \hline \hline  \endfirsthead 
\caption{Content of table msg\_log (continued)} \\ \hline \multicolumn{1}{|c|}{\textbf{msg\_id}} & \multicolumn{1}{|c|}{\textbf{time}} & \multicolumn{1}{|c|}{\textbf{ip\_address}} & \multicolumn{1}{|c|}{\textbf{subsys}} & \multicolumn{1}{|c|}{\textbf{msgs}} & \multicolumn{1}{|c|}{\textbf{type}} & \multicolumn{1}{|c|}{\textbf{is\_error}} \\ \hline \hline \endhead \endfoot
1 & 1457553874 & \textit{NULL} & \textit{NULL} & Shift status of james changed to System Manager by root. & Shifts & 0 \\ \hline 
 \end{longtable}

%
% Structure: msg_log_types
%
 \begin{longtable}{|l|c|c|c|l|} 
 \caption{Structure of table msg\_log\_types} \label{tab:msg_log_types-structure} \\
 \hline \multicolumn{1}{|c|}{\textbf{Column}} & \multicolumn{1}{|c|}{\textbf{Type}} & \multicolumn{1}{|c|}{\textbf{Null}} & \multicolumn{1}{|c|}{\textbf{Default}} & \multicolumn{1}{|c|}{\textbf{Comments}} \\ \hline \hline
\endfirsthead
 \caption{Structure of table msg\_log\_types (continued)} \\ 
 \hline \multicolumn{1}{|c|}{\textbf{Column}} & \multicolumn{1}{|c|}{\textbf{Type}} & \multicolumn{1}{|c|}{\textbf{Null}} & \multicolumn{1}{|c|}{\textbf{Default}} & \multicolumn{1}{|c|}{\textbf{Comments}} \\ \hline \hline \endhead \endfoot 
\textbf{\textit{types}} & varchar(16) & No &  \\ \hline 
 \end{longtable}

%
% Data: msg_log_types
%
 \begin{longtable}{|l|} 
 \hline \endhead \hline \endfoot \hline 
 \caption{Content of table msg\_log\_types} \label{tab:msg_log_types-data} \\\hline \multicolumn{1}{|c|}{\textbf{types}} \\ \hline \hline  \endfirsthead 
\caption{Content of table msg\_log\_types (continued)} \\ \hline \multicolumn{1}{|c|}{\textbf{types}} \\ \hline \hline \endhead \endfoot
Alarm \\ \hline 
Alert \\ \hline 
Config. \\ \hline 
Error \\ \hline 
Setpoint \\ \hline 
Shifts \\ \hline 
 \end{longtable}

%
% Structure: runs
%
 \begin{longtable}{|l|c|c|c|l|} 
 \caption{Structure of table runs} \label{tab:runs-structure} \\
 \hline \multicolumn{1}{|c|}{\textbf{Column}} & \multicolumn{1}{|c|}{\textbf{Type}} & \multicolumn{1}{|c|}{\textbf{Null}} & \multicolumn{1}{|c|}{\textbf{Default}} & \multicolumn{1}{|c|}{\textbf{Comments}} \\ \hline \hline
\endfirsthead
 \caption{Structure of table runs (continued)} \\ 
 \hline \multicolumn{1}{|c|}{\textbf{Column}} & \multicolumn{1}{|c|}{\textbf{Type}} & \multicolumn{1}{|c|}{\textbf{Null}} & \multicolumn{1}{|c|}{\textbf{Default}} & \multicolumn{1}{|c|}{\textbf{Comments}} \\ \hline \hline \endhead \endfoot 
\textbf{\textit{num}} & int(11) & No &  \\ \hline 
start\_t & int(11) & No &  \\ \hline 
end\_t & int(11) & No & 0 \\ \hline 
file\_path & varchar(128) & Yes & NULL \\ \hline 
file\_root & varchar(128) & Yes & NULL \\ \hline 
note & char(128) & Yes & NULL \\ \hline 
 \end{longtable}

%
% Data: runs
%
 \begin{longtable}{|l|l|l|l|l|l|} 
 \hline \endhead \hline \endfoot \hline 
 \caption{Content of table runs} \label{tab:runs-data} \\\hline \multicolumn{1}{|c|}{\textbf{num}} & \multicolumn{1}{|c|}{\textbf{start\_t}} & \multicolumn{1}{|c|}{\textbf{end\_t}} & \multicolumn{1}{|c|}{\textbf{file\_path}} & \multicolumn{1}{|c|}{\textbf{file\_root}} & \multicolumn{1}{|c|}{\textbf{note}} \\ \hline \hline  \endfirsthead 
\caption{Content of table runs (continued)} \\ \hline \multicolumn{1}{|c|}{\textbf{num}} & \multicolumn{1}{|c|}{\textbf{start\_t}} & \multicolumn{1}{|c|}{\textbf{end\_t}} & \multicolumn{1}{|c|}{\textbf{file\_path}} & \multicolumn{1}{|c|}{\textbf{file\_root}} & \multicolumn{1}{|c|}{\textbf{note}} \\ \hline \hline \endhead \endfoot
1 & 1342019232 & 1457624232 &  &  & Setup \\ \hline 
 \end{longtable}

%
% Structure: sc_insts
%
 \begin{longtable}{|l|c|c|c|l|} 
 \caption{Structure of table sc\_insts} \label{tab:sc_insts-structure} \\
 \hline \multicolumn{1}{|c|}{\textbf{Column}} & \multicolumn{1}{|c|}{\textbf{Type}} & \multicolumn{1}{|c|}{\textbf{Null}} & \multicolumn{1}{|c|}{\textbf{Default}} & \multicolumn{1}{|c|}{\textbf{Comments}} \\ \hline \hline
\endfirsthead
 \caption{Structure of table sc\_insts (continued)} \\ 
 \hline \multicolumn{1}{|c|}{\textbf{Column}} & \multicolumn{1}{|c|}{\textbf{Type}} & \multicolumn{1}{|c|}{\textbf{Null}} & \multicolumn{1}{|c|}{\textbf{Default}} & \multicolumn{1}{|c|}{\textbf{Comments}} \\ \hline \hline \endhead \endfoot 
\textbf{\textit{name}} & varchar(16) & No &  \\ \hline 
description & varchar(64) & Yes & NULL \\ \hline 
subsys & varchar(16) & Yes & NULL \\ \hline 
run & tinyint(1) & No & 0 \\ \hline 
restart & tinyint(1) & No & 0 \\ \hline 
WD\_ctrl & tinyint(1) & No & 1 \\ \hline 
path & varchar(256) & No & rel\_path \\ \hline 
dev\_type & varchar(16) & Yes & NULL \\ \hline 
dev\_address & varchar(24) & Yes & NULL \\ \hline 
start\_time & int(11) & No & -1 \\ \hline 
last\_update\_time & int(11) & No & -1 \\ \hline 
PID & int(11) & No & -1 \\ \hline 
user1 & varchar(16) & Yes & NULL \\ \hline 
user2 & varchar(16) & Yes & NULL \\ \hline 
parm1 & double & No & 0 \\ \hline 
parm2 & double & No & 0 \\ \hline 
notes & text & Yes & NULL \\ \hline 
 \end{longtable}

%
% Data: sc_insts
%
 \begin{longtable}{|l|l|l|l|l|l|l|l|l|l|l|l|l|l|l|l|l|} 
 \hline \endhead \hline \endfoot \hline 
 \caption{Content of table sc\_insts} \label{tab:sc_insts-data} \\\hline \multicolumn{1}{|c|}{\textbf{name}} & \multicolumn{1}{|c|}{\textbf{description}} & \multicolumn{1}{|c|}{\textbf{subsys}} & \multicolumn{1}{|c|}{\textbf{run}} & \multicolumn{1}{|c|}{\textbf{restart}} & \multicolumn{1}{|c|}{\textbf{WD\_ctrl}} & \multicolumn{1}{|c|}{\textbf{path}} & \multicolumn{1}{|c|}{\textbf{dev\_type}} & \multicolumn{1}{|c|}{\textbf{dev\_address}} & \multicolumn{1}{|c|}{\textbf{start\_time}} & \multicolumn{1}{|c|}{\textbf{last\_update\_time}} & \multicolumn{1}{|c|}{\textbf{PID}} & \multicolumn{1}{|c|}{\textbf{user1}} & \multicolumn{1}{|c|}{\textbf{user2}} & \multicolumn{1}{|c|}{\textbf{parm1}} & \multicolumn{1}{|c|}{\textbf{parm2}} & \multicolumn{1}{|c|}{\textbf{notes}} \\ \hline \hline  \endfirsthead 
\caption{Content of table sc\_insts (continued)} \\ \hline \multicolumn{1}{|c|}{\textbf{name}} & \multicolumn{1}{|c|}{\textbf{description}} & \multicolumn{1}{|c|}{\textbf{subsys}} & \multicolumn{1}{|c|}{\textbf{run}} & \multicolumn{1}{|c|}{\textbf{restart}} & \multicolumn{1}{|c|}{\textbf{WD\_ctrl}} & \multicolumn{1}{|c|}{\textbf{path}} & \multicolumn{1}{|c|}{\textbf{dev\_type}} & \multicolumn{1}{|c|}{\textbf{dev\_address}} & \multicolumn{1}{|c|}{\textbf{start\_time}} & \multicolumn{1}{|c|}{\textbf{last\_update\_time}} & \multicolumn{1}{|c|}{\textbf{PID}} & \multicolumn{1}{|c|}{\textbf{user1}} & \multicolumn{1}{|c|}{\textbf{user2}} & \multicolumn{1}{|c|}{\textbf{parm1}} & \multicolumn{1}{|c|}{\textbf{parm2}} & \multicolumn{1}{|c|}{\textbf{notes}} \\ \hline \hline \endhead \endfoot
alarm\_trip\_sys & Alarm trip monitor program & Alarm & 0 & 0 & 1 & alarm\_trip\_system/alarm\_trip\_sys & daemon &  & -1 & -1 & -1 &  &  & 0 & 0 &  \\ \hline 
alarm\_alert\_sys & Alarm alert monitor program & Alarm & 0 & 0 & 1 & alarm\_alert\_system/alarm\_alert\_sys & daemon &  & -1 & -1 & -1 & Alarm\_Light & Alarm\_Siren & 0 & 0 &  \\ \hline 
Watchdog & Overseeing watchdog program & Watchdog & 1 & 0 & 0 & /home/james/code/astro-slow-control/SC\_backend/slow\_control\_code/ & watchdog &  & 1468432011 & 1474551505 & 27010 &  &  & 0 & 0 & This should always be running.  If not, you must start it manually using the path above.  The Watchdog will start all other instruments/daemons automatically if their `run` flag is set.   \\ \hline 
alarm\_siren & Light and Siren on top of the HV rack & Alarm & 0 & 0 & 1 & ADAM6000/ADAM6060 & modbus & 192.168.91.91 & -1 & -1 & -1 & \textit{NULL} & \textit{NULL} & 0 & 0 &  \\ \hline 
disk\_free & Monitor free disk space & Sys & 1 & 0 & 1 & disk\_free/disk\_free & daemon & \textit{NULL} & 1468432011 & 1474551501 & 27149 & \textit{NULL} & \textit{NULL} & 0 & 0 &  \\ \hline 
Arduino\_IO & Arduino general IO & IO & 0 & 0 & 1 & Arduino\_ADC/Arduino\_ADC & ethernet & 192.168.1.90 & -1 & -1 & -1 & 5000 & \textit{NULL} & 0 & 0 & \textit{NULL} \\ \hline 
 \end{longtable}

%
% Structure: sc_sensors
%
 \begin{longtable}{|l|c|c|c|l|} 
 \caption{Structure of table sc\_sensors} \label{tab:sc_sensors-structure} \\
 \hline \multicolumn{1}{|c|}{\textbf{Column}} & \multicolumn{1}{|c|}{\textbf{Type}} & \multicolumn{1}{|c|}{\textbf{Null}} & \multicolumn{1}{|c|}{\textbf{Default}} & \multicolumn{1}{|c|}{\textbf{Comments}} \\ \hline \hline
\endfirsthead
 \caption{Structure of table sc\_sensors (continued)} \\ 
 \hline \multicolumn{1}{|c|}{\textbf{Column}} & \multicolumn{1}{|c|}{\textbf{Type}} & \multicolumn{1}{|c|}{\textbf{Null}} & \multicolumn{1}{|c|}{\textbf{Default}} & \multicolumn{1}{|c|}{\textbf{Comments}} \\ \hline \hline \endhead \endfoot 
\textbf{\textit{name}} & varchar(16) & No &  \\ \hline 
description & varchar(64) & Yes & NULL \\ \hline 
type & varchar(16) & No & unknown \\ \hline 
subtype & varchar(16) & Yes & NULL \\ \hline 
ctrl\_priv & varchar(128) & No & full \\ \hline 
num & int(11) & No & 0 \\ \hline 
instrument & varchar(16) & No & inst\_name \\ \hline 
units & varchar(16) & No & units \\ \hline 
discrete\_vals & varchar(128) & Yes & NULL \\ \hline 
al\_set\_val\_low & double & No & 0 \\ \hline 
al\_set\_val\_high & double & No & 0 \\ \hline 
al\_arm\_val\_low & tinyint(1) & No & 0 \\ \hline 
al\_arm\_val\_high & tinyint(1) & No & 0 \\ \hline 
al\_set\_rate\_low & double & No & 0 \\ \hline 
al\_set\_rate\_high & double & No & 0 \\ \hline 
al\_arm\_rate\_low & tinyint(1) & No & 0 \\ \hline 
al\_arm\_rate\_high & tinyint(1) & No & 0 \\ \hline 
alarm\_tripped & tinyint(1) & No & 0 \\ \hline 
grace & int(11) & No & 0 \\ \hline 
last\_trip & int(11) & No & -1 \\ \hline 
settable & tinyint(1) & No & 0 \\ \hline 
show\_rate & tinyint(1) & No & 0 \\ \hline 
hide\_sensor & tinyint(1) & No & 0 \\ \hline 
update\_period & int(11) & No & 60 \\ \hline 
num\_format & varchar(16) & Yes & NULL \\ \hline 
user1 & varchar(16) & Yes & NULL \\ \hline 
user2 & varchar(16) & Yes & NULL \\ \hline 
user3 & varchar(16) & Yes & NULL \\ \hline 
user4 & varchar(16) & Yes & NULL \\ \hline 
parm1 & double & No & 0 \\ \hline 
parm2 & double & No & 0 \\ \hline 
parm3 & double & No & 0 \\ \hline 
parm4 & double & No & 0 \\ \hline 
notes & text & Yes & NULL \\ \hline 
 \end{longtable}

%
% Data: sc_sensors
%
 \begin{longtable}{|l|l|l|l|l|l|l|l|l|l|l|l|l|l|l|l|l|l|l|l|l|l|l|l|l|l|l|l|l|l|l|l|l|l|} 
 \hline \endhead \hline \endfoot \hline 
 \caption{Content of table sc\_sensors} \label{tab:sc_sensors-data} \\\hline \multicolumn{1}{|c|}{\textbf{name}} & \multicolumn{1}{|c|}{\textbf{description}} & \multicolumn{1}{|c|}{\textbf{type}} & \multicolumn{1}{|c|}{\textbf{subtype}} & \multicolumn{1}{|c|}{\textbf{ctrl\_priv}} & \multicolumn{1}{|c|}{\textbf{num}} & \multicolumn{1}{|c|}{\textbf{instrument}} & \multicolumn{1}{|c|}{\textbf{units}} & \multicolumn{1}{|c|}{\textbf{discrete\_vals}} & \multicolumn{1}{|c|}{\textbf{al\_set\_val\_low}} & \multicolumn{1}{|c|}{\textbf{al\_set\_val\_high}} & \multicolumn{1}{|c|}{\textbf{al\_arm\_val\_low}} & \multicolumn{1}{|c|}{\textbf{al\_arm\_val\_high}} & \multicolumn{1}{|c|}{\textbf{al\_set\_rate\_low}} & \multicolumn{1}{|c|}{\textbf{al\_set\_rate\_high}} & \multicolumn{1}{|c|}{\textbf{al\_arm\_rate\_low}} & \multicolumn{1}{|c|}{\textbf{al\_arm\_rate\_high}} & \multicolumn{1}{|c|}{\textbf{alarm\_tripped}} & \multicolumn{1}{|c|}{\textbf{grace}} & \multicolumn{1}{|c|}{\textbf{last\_trip}} & \multicolumn{1}{|c|}{\textbf{settable}} & \multicolumn{1}{|c|}{\textbf{show\_rate}} & \multicolumn{1}{|c|}{\textbf{hide\_sensor}} & \multicolumn{1}{|c|}{\textbf{update\_period}} & \multicolumn{1}{|c|}{\textbf{num\_format}} & \multicolumn{1}{|c|}{\textbf{user1}} & \multicolumn{1}{|c|}{\textbf{user2}} & \multicolumn{1}{|c|}{\textbf{user3}} & \multicolumn{1}{|c|}{\textbf{user4}} & \multicolumn{1}{|c|}{\textbf{parm1}} & \multicolumn{1}{|c|}{\textbf{parm2}} & \multicolumn{1}{|c|}{\textbf{parm3}} & \multicolumn{1}{|c|}{\textbf{parm4}} & \multicolumn{1}{|c|}{\textbf{notes}} \\ \hline \hline  \endfirsthead 
\caption{Content of table sc\_sensors (continued)} \\ \hline \multicolumn{1}{|c|}{\textbf{name}} & \multicolumn{1}{|c|}{\textbf{description}} & \multicolumn{1}{|c|}{\textbf{type}} & \multicolumn{1}{|c|}{\textbf{subtype}} & \multicolumn{1}{|c|}{\textbf{ctrl\_priv}} & \multicolumn{1}{|c|}{\textbf{num}} & \multicolumn{1}{|c|}{\textbf{instrument}} & \multicolumn{1}{|c|}{\textbf{units}} & \multicolumn{1}{|c|}{\textbf{discrete\_vals}} & \multicolumn{1}{|c|}{\textbf{al\_set\_val\_low}} & \multicolumn{1}{|c|}{\textbf{al\_set\_val\_high}} & \multicolumn{1}{|c|}{\textbf{al\_arm\_val\_low}} & \multicolumn{1}{|c|}{\textbf{al\_arm\_val\_high}} & \multicolumn{1}{|c|}{\textbf{al\_set\_rate\_low}} & \multicolumn{1}{|c|}{\textbf{al\_set\_rate\_high}} & \multicolumn{1}{|c|}{\textbf{al\_arm\_rate\_low}} & \multicolumn{1}{|c|}{\textbf{al\_arm\_rate\_high}} & \multicolumn{1}{|c|}{\textbf{alarm\_tripped}} & \multicolumn{1}{|c|}{\textbf{grace}} & \multicolumn{1}{|c|}{\textbf{last\_trip}} & \multicolumn{1}{|c|}{\textbf{settable}} & \multicolumn{1}{|c|}{\textbf{show\_rate}} & \multicolumn{1}{|c|}{\textbf{hide\_sensor}} & \multicolumn{1}{|c|}{\textbf{update\_period}} & \multicolumn{1}{|c|}{\textbf{num\_format}} & \multicolumn{1}{|c|}{\textbf{user1}} & \multicolumn{1}{|c|}{\textbf{user2}} & \multicolumn{1}{|c|}{\textbf{user3}} & \multicolumn{1}{|c|}{\textbf{user4}} & \multicolumn{1}{|c|}{\textbf{parm1}} & \multicolumn{1}{|c|}{\textbf{parm2}} & \multicolumn{1}{|c|}{\textbf{parm3}} & \multicolumn{1}{|c|}{\textbf{parm4}} & \multicolumn{1}{|c|}{\textbf{notes}} \\ \hline \hline \endhead \endfoot
Alarm\_Light & Alarm light on the HV rack & 1 & Relay & siren & 0 & alarm\_siren & discrete & 0:1;Off:On & 0 & 0 & 0 & 0 & 0 & 0 & 0 & 0 & 0 & 0 & -1 & 1 & 0 & 0 & 60 & \textit{NULL} & \textit{NULL} & \textit{NULL} & \textit{NULL} & \textit{NULL} & 1 & 0 & 0 & 0 &  \\ \hline 
Alarm\_Siren & Alarm siren on the HV rack & 1 & Relay & siren & 1 & alarm\_siren & discrete & 0:1;Off:On & 0 & 0 & 0 & 0 & 0 & 0 & 0 & 0 & 0 & 0 & -1 & 1 & 0 & 0 & 60 & \textit{NULL} & \textit{NULL} & \textit{NULL} & \textit{NULL} & \textit{NULL} & 1 & 0 & 0 & 0 &  \\ \hline 
dsk\_shm & Free space on /dev/shm & 2 & disk & full & 0 & disk\_free & \% & \textit{NULL} & 50 & 99 & 1 & 0 & 0 & 0 & 0 & 0 & 0 & 0 & -1 & 0 & 0 & 0 & 30 & \textit{NULL} & /dev/shm & \textit{NULL} & \textit{NULL} & \textit{NULL} & 0 & 0 & 0 & 0 &  \\ \hline 
dsk\_root & Free space on / & 2 & disk & full & 0 & disk\_free & \% & \textit{NULL} & 30 & 70 & 1 & 0 & 0 & 0 & 0 & 0 & 0 & 0 & -1 & 0 & 0 & 0 & 30 & \textit{NULL} & / & \textit{NULL} & \textit{NULL} & \textit{NULL} & 0 & 0 & 0 & 0 &  \\ \hline 
Set\_Temp\_A & \textit{NULL} & 3 & \textit{NULL} & full,TempControl & 0 & inst\_name & K & \textit{NULL} & 0 & 0 & 0 & 0 & 0 & 0 & 0 & 0 & 0 & 0 & -1 & 1 & 0 & 1 & 60 & \textit{NULL} & \textit{NULL} & \textit{NULL} & \textit{NULL} & \textit{NULL} & 0 & 0 & 0 & 0 & \textit{NULL} \\ \hline 
Temp\_A & \textit{NULL} & 3 & \textit{NULL} & full & 0 & inst\_name & K & \textit{NULL} & 0 & 0 & 0 & 0 & 0 & 0 & 0 & 0 & 0 & 0 & -1 & 0 & 1 & 1 & 60 & \textit{NULL} & \textit{NULL} & \textit{NULL} & \textit{NULL} & \textit{NULL} & 0 & 0 & 0 & 0 & \textit{NULL} \\ \hline 
Temp\_B & \textit{NULL} & 3 & \textit{NULL} & full & 1 & inst\_name & K & \textit{NULL} & 0 & 0 & 0 & 0 & 0 & 0 & 0 & 0 & 0 & 0 & -1 & 0 & 0 & 1 & 60 & \textit{NULL} & \textit{NULL} & \textit{NULL} & \textit{NULL} & \textit{NULL} & 0 & 0 & 0 & 0 & \textit{NULL} \\ \hline 
TC\_onoff & \textit{NULL} & 3 & \textit{NULL} & full,TempControl & 0 & inst\_name & discrete & 0:1;Off:On & 0 & 0 & 0 & 0 & 0 & 0 & 0 & 0 & 0 & 0 & -1 & 1 & 0 & 0 & 60 & \textit{NULL} & \textit{NULL} & \textit{NULL} & \textit{NULL} & \textit{NULL} & 0 & 0 & 0 & 0 & \textit{NULL} \\ \hline 
Run\_ctrl & Run control handle & 6 & ctrl & full & 0 & inst\_name & discrete & 0:1;Off:On & 0 & 0 & 0 & 0 & 0 & 0 & 0 & 0 & 0 & 0 & -1 & 1 & 0 & 0 & 60 & \textit{NULL} & \textit{NULL} & \textit{NULL} & \textit{NULL} & \textit{NULL} & 0 & 0 & 0 & 0 & \textit{NULL} \\ \hline 
 \end{longtable}

%
% Structure: sc_sensor_types
%
 \begin{longtable}{|l|c|c|c|l|} 
 \caption{Structure of table sc\_sensor\_types} \label{tab:sc_sensor_types-structure} \\
 \hline \multicolumn{1}{|c|}{\textbf{Column}} & \multicolumn{1}{|c|}{\textbf{Type}} & \multicolumn{1}{|c|}{\textbf{Null}} & \multicolumn{1}{|c|}{\textbf{Default}} & \multicolumn{1}{|c|}{\textbf{Comments}} \\ \hline \hline
\endfirsthead
 \caption{Structure of table sc\_sensor\_types (continued)} \\ 
 \hline \multicolumn{1}{|c|}{\textbf{Column}} & \multicolumn{1}{|c|}{\textbf{Type}} & \multicolumn{1}{|c|}{\textbf{Null}} & \multicolumn{1}{|c|}{\textbf{Default}} & \multicolumn{1}{|c|}{\textbf{Comments}} \\ \hline \hline \endhead \endfoot 
num & int(11) & No &  \\ \hline 
\textbf{\textit{name}} & varchar(16) & No &  \\ \hline 
 \end{longtable}

%
% Data: sc_sensor_types
%
 \begin{longtable}{|l|l|} 
 \hline \endhead \hline \endfoot \hline 
 \caption{Content of table sc\_sensor\_types} \label{tab:sc_sensor_types-data} \\\hline \multicolumn{1}{|c|}{\textbf{num}} & \multicolumn{1}{|c|}{\textbf{name}} \\ \hline \hline  \endfirsthead 
\caption{Content of table sc\_sensor\_types (continued)} \\ \hline \multicolumn{1}{|c|}{\textbf{num}} & \multicolumn{1}{|c|}{\textbf{name}} \\ \hline \hline \endhead \endfoot
1 & Alarm \\ \hline 
2 & Sys \\ \hline 
3 & Temperature \\ \hline 
6 & DAQ \\ \hline 
7 & TS \\ \hline 
 \end{longtable}

%
% Structure: sc_sens_Alarm_Light
%
 \begin{longtable}{|l|c|c|c|l|} 
 \caption{Structure of table sc\_sens\_Alarm\_Light} \label{tab:sc_sens_Alarm_Light-structure} \\
 \hline \multicolumn{1}{|c|}{\textbf{Column}} & \multicolumn{1}{|c|}{\textbf{Type}} & \multicolumn{1}{|c|}{\textbf{Null}} & \multicolumn{1}{|c|}{\textbf{Default}} & \multicolumn{1}{|c|}{\textbf{Comments}} \\ \hline \hline
\endfirsthead
 \caption{Structure of table sc\_sens\_Alarm\_Light (continued)} \\ 
 \hline \multicolumn{1}{|c|}{\textbf{Column}} & \multicolumn{1}{|c|}{\textbf{Type}} & \multicolumn{1}{|c|}{\textbf{Null}} & \multicolumn{1}{|c|}{\textbf{Default}} & \multicolumn{1}{|c|}{\textbf{Comments}} \\ \hline \hline \endhead \endfoot 
time & int(11) & No &  \\ \hline 
value & double & No &  \\ \hline 
rate & double & No &  \\ \hline 
 \end{longtable}

%
% Data: sc_sens_Alarm_Light
%
 \begin{longtable}{|l|l|l|} 
 \hline \endhead \hline \endfoot \hline 
 \caption{Content of table sc\_sens\_Alarm\_Light} \label{tab:sc_sens_Alarm_Light-data} \\\hline \multicolumn{1}{|c|}{\textbf{time}} & \multicolumn{1}{|c|}{\textbf{value}} & \multicolumn{1}{|c|}{\textbf{rate}} \\ \hline \hline  \endfirsthead 
\caption{Content of table sc\_sens\_Alarm\_Light (continued)} \\ \hline \multicolumn{1}{|c|}{\textbf{time}} & \multicolumn{1}{|c|}{\textbf{value}} & \multicolumn{1}{|c|}{\textbf{rate}} \\ \hline \hline \endhead \endfoot
1457555287 & 2.3 & 0 \\ \hline 
1459350469 & 0 & 0 \\ \hline 
 \end{longtable}

%
% Structure: sc_sens_Alarm_Siren
%
 \begin{longtable}{|l|c|c|c|l|} 
 \caption{Structure of table sc\_sens\_Alarm\_Siren} \label{tab:sc_sens_Alarm_Siren-structure} \\
 \hline \multicolumn{1}{|c|}{\textbf{Column}} & \multicolumn{1}{|c|}{\textbf{Type}} & \multicolumn{1}{|c|}{\textbf{Null}} & \multicolumn{1}{|c|}{\textbf{Default}} & \multicolumn{1}{|c|}{\textbf{Comments}} \\ \hline \hline
\endfirsthead
 \caption{Structure of table sc\_sens\_Alarm\_Siren (continued)} \\ 
 \hline \multicolumn{1}{|c|}{\textbf{Column}} & \multicolumn{1}{|c|}{\textbf{Type}} & \multicolumn{1}{|c|}{\textbf{Null}} & \multicolumn{1}{|c|}{\textbf{Default}} & \multicolumn{1}{|c|}{\textbf{Comments}} \\ \hline \hline \endhead \endfoot 
time & int(11) & No &  \\ \hline 
value & double & No &  \\ \hline 
rate & double & No &  \\ \hline 
 \end{longtable}

%
% Data: sc_sens_Alarm_Siren
%
 \begin{longtable}{|l|l|l|} 
 \hline \endhead \hline \endfoot \hline 
 \caption{Content of table sc\_sens\_Alarm\_Siren} \label{tab:sc_sens_Alarm_Siren-data} \\\hline \multicolumn{1}{|c|}{\textbf{time}} & \multicolumn{1}{|c|}{\textbf{value}} & \multicolumn{1}{|c|}{\textbf{rate}} \\ \hline \hline  \endfirsthead 
\caption{Content of table sc\_sens\_Alarm\_Siren (continued)} \\ \hline \multicolumn{1}{|c|}{\textbf{time}} & \multicolumn{1}{|c|}{\textbf{value}} & \multicolumn{1}{|c|}{\textbf{rate}} \\ \hline \hline \endhead \endfoot
1459350470 & 0 & 0 \\ \hline 
 \end{longtable}

%
% Structure: sc_sens_DAQ_ctrl
%
 \begin{longtable}{|l|c|c|c|l|} 
 \caption{Structure of table sc\_sens\_DAQ\_ctrl} \label{tab:sc_sens_DAQ_ctrl-structure} \\
 \hline \multicolumn{1}{|c|}{\textbf{Column}} & \multicolumn{1}{|c|}{\textbf{Type}} & \multicolumn{1}{|c|}{\textbf{Null}} & \multicolumn{1}{|c|}{\textbf{Default}} & \multicolumn{1}{|c|}{\textbf{Comments}} \\ \hline \hline
\endfirsthead
 \caption{Structure of table sc\_sens\_DAQ\_ctrl (continued)} \\ 
 \hline \multicolumn{1}{|c|}{\textbf{Column}} & \multicolumn{1}{|c|}{\textbf{Type}} & \multicolumn{1}{|c|}{\textbf{Null}} & \multicolumn{1}{|c|}{\textbf{Default}} & \multicolumn{1}{|c|}{\textbf{Comments}} \\ \hline \hline \endhead \endfoot 
time & int(11) & No &  \\ \hline 
value & double & No &  \\ \hline 
rate & double & No &  \\ \hline 
 \end{longtable}

%
% Data: sc_sens_DAQ_ctrl
%
 \begin{longtable}{|l|l|l|} 
 \hline \endhead \hline \endfoot \hline 
 \caption{Content of table sc\_sens\_DAQ\_ctrl} \label{tab:sc_sens_DAQ_ctrl-data} \\\hline \multicolumn{1}{|c|}{\textbf{time}} & \multicolumn{1}{|c|}{\textbf{value}} & \multicolumn{1}{|c|}{\textbf{rate}} \\ \hline \hline  \endfirsthead 
\caption{Content of table sc\_sens\_DAQ\_ctrl (continued)} \\ \hline \multicolumn{1}{|c|}{\textbf{time}} & \multicolumn{1}{|c|}{\textbf{value}} & \multicolumn{1}{|c|}{\textbf{rate}} \\ \hline \hline \endhead \endfoot
1461529514 & 1 & 0 \\ \hline 
1461529515 & 0 & 0 \\ \hline 
 \end{longtable}

%
% Structure: sc_sens_dsk_root
%
 \begin{longtable}{|l|c|c|c|l|} 
 \caption{Structure of table sc\_sens\_dsk\_root} \label{tab:sc_sens_dsk_root-structure} \\
 \hline \multicolumn{1}{|c|}{\textbf{Column}} & \multicolumn{1}{|c|}{\textbf{Type}} & \multicolumn{1}{|c|}{\textbf{Null}} & \multicolumn{1}{|c|}{\textbf{Default}} & \multicolumn{1}{|c|}{\textbf{Comments}} \\ \hline \hline
\endfirsthead
 \caption{Structure of table sc\_sens\_dsk\_root (continued)} \\ 
 \hline \multicolumn{1}{|c|}{\textbf{Column}} & \multicolumn{1}{|c|}{\textbf{Type}} & \multicolumn{1}{|c|}{\textbf{Null}} & \multicolumn{1}{|c|}{\textbf{Default}} & \multicolumn{1}{|c|}{\textbf{Comments}} \\ \hline \hline \endhead \endfoot 
time & int(11) & No &  \\ \hline 
value & double & No &  \\ \hline 
rate & double & No &  \\ \hline 
 \end{longtable}

%
% Data: sc_sens_dsk_root
%
 \begin{longtable}{|l|l|l|} 
 \hline \endhead \hline \endfoot \hline 
 \caption{Content of table sc\_sens\_dsk\_root} \label{tab:sc_sens_dsk_root-data} \\\hline \multicolumn{1}{|c|}{\textbf{time}} & \multicolumn{1}{|c|}{\textbf{value}} & \multicolumn{1}{|c|}{\textbf{rate}} \\ \hline \hline  \endfirsthead 
\caption{Content of table sc\_sens\_dsk\_root (continued)} \\ \hline \multicolumn{1}{|c|}{\textbf{time}} & \multicolumn{1}{|c|}{\textbf{value}} & \multicolumn{1}{|c|}{\textbf{rate}} \\ \hline \hline \endhead \endfoot
 \end{longtable}

%
% Structure: sc_sens_dsk_shm
%
 \begin{longtable}{|l|c|c|c|l|} 
 \caption{Structure of table sc\_sens\_dsk\_shm} \label{tab:sc_sens_dsk_shm-structure} \\
 \hline \multicolumn{1}{|c|}{\textbf{Column}} & \multicolumn{1}{|c|}{\textbf{Type}} & \multicolumn{1}{|c|}{\textbf{Null}} & \multicolumn{1}{|c|}{\textbf{Default}} & \multicolumn{1}{|c|}{\textbf{Comments}} \\ \hline \hline
\endfirsthead
 \caption{Structure of table sc\_sens\_dsk\_shm (continued)} \\ 
 \hline \multicolumn{1}{|c|}{\textbf{Column}} & \multicolumn{1}{|c|}{\textbf{Type}} & \multicolumn{1}{|c|}{\textbf{Null}} & \multicolumn{1}{|c|}{\textbf{Default}} & \multicolumn{1}{|c|}{\textbf{Comments}} \\ \hline \hline \endhead \endfoot 
time & int(11) & No &  \\ \hline 
value & double & No &  \\ \hline 
rate & double & No &  \\ \hline 
 \end{longtable}

%
% Data: sc_sens_dsk_shm
%
 \begin{longtable}{|l|l|l|} 
 \hline \endhead \hline \endfoot \hline 
 \caption{Content of table sc\_sens\_dsk\_shm} \label{tab:sc_sens_dsk_shm-data} \\\hline \multicolumn{1}{|c|}{\textbf{time}} & \multicolumn{1}{|c|}{\textbf{value}} & \multicolumn{1}{|c|}{\textbf{rate}} \\ \hline \hline  \endfirsthead 
\caption{Content of table sc\_sens\_dsk\_shm (continued)} \\ \hline \multicolumn{1}{|c|}{\textbf{time}} & \multicolumn{1}{|c|}{\textbf{value}} & \multicolumn{1}{|c|}{\textbf{rate}} \\ \hline \hline \endhead \endfoot
 \end{longtable}

%
% Structure: sc_sens_Run_ctrl
%
 \begin{longtable}{|l|c|c|c|l|} 
 \caption{Structure of table sc\_sens\_Run\_ctrl} \label{tab:sc_sens_Run_ctrl-structure} \\
 \hline \multicolumn{1}{|c|}{\textbf{Column}} & \multicolumn{1}{|c|}{\textbf{Type}} & \multicolumn{1}{|c|}{\textbf{Null}} & \multicolumn{1}{|c|}{\textbf{Default}} & \multicolumn{1}{|c|}{\textbf{Comments}} \\ \hline \hline
\endfirsthead
 \caption{Structure of table sc\_sens\_Run\_ctrl (continued)} \\ 
 \hline \multicolumn{1}{|c|}{\textbf{Column}} & \multicolumn{1}{|c|}{\textbf{Type}} & \multicolumn{1}{|c|}{\textbf{Null}} & \multicolumn{1}{|c|}{\textbf{Default}} & \multicolumn{1}{|c|}{\textbf{Comments}} \\ \hline \hline \endhead \endfoot 
time & int(11) & No &  \\ \hline 
value & double & No &  \\ \hline 
rate & double & No &  \\ \hline 
 \end{longtable}

%
% Data: sc_sens_Run_ctrl
%
 \begin{longtable}{|l|l|l|} 
 \hline \endhead \hline \endfoot \hline 
 \caption{Content of table sc\_sens\_Run\_ctrl} \label{tab:sc_sens_Run_ctrl-data} \\\hline \multicolumn{1}{|c|}{\textbf{time}} & \multicolumn{1}{|c|}{\textbf{value}} & \multicolumn{1}{|c|}{\textbf{rate}} \\ \hline \hline  \endfirsthead 
\caption{Content of table sc\_sens\_Run\_ctrl (continued)} \\ \hline \multicolumn{1}{|c|}{\textbf{time}} & \multicolumn{1}{|c|}{\textbf{value}} & \multicolumn{1}{|c|}{\textbf{rate}} \\ \hline \hline \endhead \endfoot
1461529604 & 1 & 0 \\ \hline 
1461529605 & 0 & 0 \\ \hline 
 \end{longtable}

%
% Structure: sc_sens_Set_Temp_A
%
 \begin{longtable}{|l|c|c|c|l|} 
 \caption{Structure of table sc\_sens\_Set\_Temp\_A} \label{tab:sc_sens_Set_Temp_A-structure} \\
 \hline \multicolumn{1}{|c|}{\textbf{Column}} & \multicolumn{1}{|c|}{\textbf{Type}} & \multicolumn{1}{|c|}{\textbf{Null}} & \multicolumn{1}{|c|}{\textbf{Default}} & \multicolumn{1}{|c|}{\textbf{Comments}} \\ \hline \hline
\endfirsthead
 \caption{Structure of table sc\_sens\_Set\_Temp\_A (continued)} \\ 
 \hline \multicolumn{1}{|c|}{\textbf{Column}} & \multicolumn{1}{|c|}{\textbf{Type}} & \multicolumn{1}{|c|}{\textbf{Null}} & \multicolumn{1}{|c|}{\textbf{Default}} & \multicolumn{1}{|c|}{\textbf{Comments}} \\ \hline \hline \endhead \endfoot 
time & int(11) & No &  \\ \hline 
value & double & No &  \\ \hline 
rate & double & No &  \\ \hline 
 \end{longtable}

%
% Data: sc_sens_Set_Temp_A
%
 \begin{longtable}{|l|l|l|} 
 \hline \endhead \hline \endfoot \hline 
 \caption{Content of table sc\_sens\_Set\_Temp\_A} \label{tab:sc_sens_Set_Temp_A-data} \\\hline \multicolumn{1}{|c|}{\textbf{time}} & \multicolumn{1}{|c|}{\textbf{value}} & \multicolumn{1}{|c|}{\textbf{rate}} \\ \hline \hline  \endfirsthead 
\caption{Content of table sc\_sens\_Set\_Temp\_A (continued)} \\ \hline \multicolumn{1}{|c|}{\textbf{time}} & \multicolumn{1}{|c|}{\textbf{value}} & \multicolumn{1}{|c|}{\textbf{rate}} \\ \hline \hline \endhead \endfoot
1457977153 & 23 & 0 \\ \hline 
 \end{longtable}

%
% Structure: sc_sens_TC_onoff
%
 \begin{longtable}{|l|c|c|c|l|} 
 \caption{Structure of table sc\_sens\_TC\_onoff} \label{tab:sc_sens_TC_onoff-structure} \\
 \hline \multicolumn{1}{|c|}{\textbf{Column}} & \multicolumn{1}{|c|}{\textbf{Type}} & \multicolumn{1}{|c|}{\textbf{Null}} & \multicolumn{1}{|c|}{\textbf{Default}} & \multicolumn{1}{|c|}{\textbf{Comments}} \\ \hline \hline
\endfirsthead
 \caption{Structure of table sc\_sens\_TC\_onoff (continued)} \\ 
 \hline \multicolumn{1}{|c|}{\textbf{Column}} & \multicolumn{1}{|c|}{\textbf{Type}} & \multicolumn{1}{|c|}{\textbf{Null}} & \multicolumn{1}{|c|}{\textbf{Default}} & \multicolumn{1}{|c|}{\textbf{Comments}} \\ \hline \hline \endhead \endfoot 
time & int(11) & No &  \\ \hline 
value & double & No &  \\ \hline 
rate & double & No &  \\ \hline 
 \end{longtable}

%
% Data: sc_sens_TC_onoff
%
 \begin{longtable}{|l|l|l|} 
 \hline \endhead \hline \endfoot \hline 
 \caption{Content of table sc\_sens\_TC\_onoff} \label{tab:sc_sens_TC_onoff-data} \\\hline \multicolumn{1}{|c|}{\textbf{time}} & \multicolumn{1}{|c|}{\textbf{value}} & \multicolumn{1}{|c|}{\textbf{rate}} \\ \hline \hline  \endfirsthead 
\caption{Content of table sc\_sens\_TC\_onoff (continued)} \\ \hline \multicolumn{1}{|c|}{\textbf{time}} & \multicolumn{1}{|c|}{\textbf{value}} & \multicolumn{1}{|c|}{\textbf{rate}} \\ \hline \hline \endhead \endfoot
1457977139 & 1 & 0 \\ \hline 
1457977151 & 0 & 0 \\ \hline 
 \end{longtable}

%
% Structure: sc_sens_Temp_A
%
 \begin{longtable}{|l|c|c|c|l|} 
 \caption{Structure of table sc\_sens\_Temp\_A} \label{tab:sc_sens_Temp_A-structure} \\
 \hline \multicolumn{1}{|c|}{\textbf{Column}} & \multicolumn{1}{|c|}{\textbf{Type}} & \multicolumn{1}{|c|}{\textbf{Null}} & \multicolumn{1}{|c|}{\textbf{Default}} & \multicolumn{1}{|c|}{\textbf{Comments}} \\ \hline \hline
\endfirsthead
 \caption{Structure of table sc\_sens\_Temp\_A (continued)} \\ 
 \hline \multicolumn{1}{|c|}{\textbf{Column}} & \multicolumn{1}{|c|}{\textbf{Type}} & \multicolumn{1}{|c|}{\textbf{Null}} & \multicolumn{1}{|c|}{\textbf{Default}} & \multicolumn{1}{|c|}{\textbf{Comments}} \\ \hline \hline \endhead \endfoot 
time & int(11) & No &  \\ \hline 
value & double & No &  \\ \hline 
rate & double & No &  \\ \hline 
 \end{longtable}

%
% Data: sc_sens_Temp_A
%
 \begin{longtable}{|l|l|l|} 
 \hline \endhead \hline \endfoot \hline 
 \caption{Content of table sc\_sens\_Temp\_A} \label{tab:sc_sens_Temp_A-data} \\\hline \multicolumn{1}{|c|}{\textbf{time}} & \multicolumn{1}{|c|}{\textbf{value}} & \multicolumn{1}{|c|}{\textbf{rate}} \\ \hline \hline  \endfirsthead 
\caption{Content of table sc\_sens\_Temp\_A (continued)} \\ \hline \multicolumn{1}{|c|}{\textbf{time}} & \multicolumn{1}{|c|}{\textbf{value}} & \multicolumn{1}{|c|}{\textbf{rate}} \\ \hline \hline \endhead \endfoot
 \end{longtable}

%
% Structure: sc_sens_Temp_B
%
 \begin{longtable}{|l|c|c|c|l|} 
 \caption{Structure of table sc\_sens\_Temp\_B} \label{tab:sc_sens_Temp_B-structure} \\
 \hline \multicolumn{1}{|c|}{\textbf{Column}} & \multicolumn{1}{|c|}{\textbf{Type}} & \multicolumn{1}{|c|}{\textbf{Null}} & \multicolumn{1}{|c|}{\textbf{Default}} & \multicolumn{1}{|c|}{\textbf{Comments}} \\ \hline \hline
\endfirsthead
 \caption{Structure of table sc\_sens\_Temp\_B (continued)} \\ 
 \hline \multicolumn{1}{|c|}{\textbf{Column}} & \multicolumn{1}{|c|}{\textbf{Type}} & \multicolumn{1}{|c|}{\textbf{Null}} & \multicolumn{1}{|c|}{\textbf{Default}} & \multicolumn{1}{|c|}{\textbf{Comments}} \\ \hline \hline \endhead \endfoot 
time & int(11) & No &  \\ \hline 
value & double & No &  \\ \hline 
rate & double & No &  \\ \hline 
 \end{longtable}

%
% Data: sc_sens_Temp_B
%
 \begin{longtable}{|l|l|l|} 
 \hline \endhead \hline \endfoot \hline 
 \caption{Content of table sc\_sens\_Temp\_B} \label{tab:sc_sens_Temp_B-data} \\\hline \multicolumn{1}{|c|}{\textbf{time}} & \multicolumn{1}{|c|}{\textbf{value}} & \multicolumn{1}{|c|}{\textbf{rate}} \\ \hline \hline  \endfirsthead 
\caption{Content of table sc\_sens\_Temp\_B (continued)} \\ \hline \multicolumn{1}{|c|}{\textbf{time}} & \multicolumn{1}{|c|}{\textbf{value}} & \multicolumn{1}{|c|}{\textbf{rate}} \\ \hline \hline \endhead \endfoot
 \end{longtable}

%
% Structure: users
%
 \begin{longtable}{|l|c|c|c|l|} 
 \caption{Structure of table users} \label{tab:users-structure} \\
 \hline \multicolumn{1}{|c|}{\textbf{Column}} & \multicolumn{1}{|c|}{\textbf{Type}} & \multicolumn{1}{|c|}{\textbf{Null}} & \multicolumn{1}{|c|}{\textbf{Default}} & \multicolumn{1}{|c|}{\textbf{Comments}} \\ \hline \hline
\endfirsthead
 \caption{Structure of table users (continued)} \\ 
 \hline \multicolumn{1}{|c|}{\textbf{Column}} & \multicolumn{1}{|c|}{\textbf{Type}} & \multicolumn{1}{|c|}{\textbf{Null}} & \multicolumn{1}{|c|}{\textbf{Default}} & \multicolumn{1}{|c|}{\textbf{Comments}} \\ \hline \hline \endhead \endfoot 
\textbf{\textit{user\_name}} & varchar(32) & No &  \\ \hline 
password & char(32) & Yes & NULL \\ \hline 
full\_name & varchar(32) & Yes & NULL \\ \hline 
affiliation & varchar(64) & Yes & NULL \\ \hline 
email & varchar(32) & Yes & NULL \\ \hline 
sms & varchar(32) & Yes & NULL \\ \hline 
phone & varchar(16) & Yes & NULL \\ \hline 
on\_call & tinyint(1) & No & 0 \\ \hline 
shift\_status & varchar(16) & Yes & NULL \\ \hline 
privileges & text & Yes & NULL \\ \hline 
 \end{longtable}

%
% Data: users
%
 \begin{longtable}{|l|l|l|l|l|l|l|l|l|l|} 
 \hline \endhead \hline \endfoot \hline 
 \caption{Content of table users} \label{tab:users-data} \\\hline \multicolumn{1}{|c|}{\textbf{user\_name}} & \multicolumn{1}{|c|}{\textbf{password}} & \multicolumn{1}{|c|}{\textbf{full\_name}} & \multicolumn{1}{|c|}{\textbf{affiliation}} & \multicolumn{1}{|c|}{\textbf{email}} & \multicolumn{1}{|c|}{\textbf{sms}} & \multicolumn{1}{|c|}{\textbf{phone}} & \multicolumn{1}{|c|}{\textbf{on\_call}} & \multicolumn{1}{|c|}{\textbf{shift\_status}} & \multicolumn{1}{|c|}{\textbf{privileges}} \\ \hline \hline  \endfirsthead 
\caption{Content of table users (continued)} \\ \hline \multicolumn{1}{|c|}{\textbf{user\_name}} & \multicolumn{1}{|c|}{\textbf{password}} & \multicolumn{1}{|c|}{\textbf{full\_name}} & \multicolumn{1}{|c|}{\textbf{affiliation}} & \multicolumn{1}{|c|}{\textbf{email}} & \multicolumn{1}{|c|}{\textbf{sms}} & \multicolumn{1}{|c|}{\textbf{phone}} & \multicolumn{1}{|c|}{\textbf{on\_call}} & \multicolumn{1}{|c|}{\textbf{shift\_status}} & \multicolumn{1}{|c|}{\textbf{privileges}} \\ \hline \hline \endhead \endfoot
guest & 084e0343a0486ff05530df6c705c8bb4 & guest & none &  &  &  & 0 & Off & basic,full,guest \\ \hline 
root & 63a9f0ea7bb98050796b649e85481845 & root & root & root & root & root & 0 & Off & admin,basic,config,full,lug,siren \\ \hline 
james & bfd59291e825b5f2bbf1eb76569f8fe7 & James Nikkel & Yale & james.nikkel@yale.edu & \textit{NULL} & 203 430 3404 & 0 & Shift Leader & admin,basic,config,DAQ,full,siren,TempControl,TS \\ \hline 
 \end{longtable}

%
% Structure: user_privileges
%
 \begin{longtable}{|l|c|c|c|l|} 
 \caption{Structure of table user\_privileges} \label{tab:user_privileges-structure} \\
 \hline \multicolumn{1}{|c|}{\textbf{Column}} & \multicolumn{1}{|c|}{\textbf{Type}} & \multicolumn{1}{|c|}{\textbf{Null}} & \multicolumn{1}{|c|}{\textbf{Default}} & \multicolumn{1}{|c|}{\textbf{Comments}} \\ \hline \hline
\endfirsthead
 \caption{Structure of table user\_privileges (continued)} \\ 
 \hline \multicolumn{1}{|c|}{\textbf{Column}} & \multicolumn{1}{|c|}{\textbf{Type}} & \multicolumn{1}{|c|}{\textbf{Null}} & \multicolumn{1}{|c|}{\textbf{Default}} & \multicolumn{1}{|c|}{\textbf{Comments}} \\ \hline \hline \endhead \endfoot 
\textbf{\textit{u\_p\_indx}} & int(11) & No &  \\ \hline 
name & varchar(16) & No &  \\ \hline 
allowed\_host & varchar(16) & No & all \\ \hline 
 \end{longtable}

%
% Data: user_privileges
%
 \begin{longtable}{|l|l|l|} 
 \hline \endhead \hline \endfoot \hline 
 \caption{Content of table user\_privileges} \label{tab:user_privileges-data} \\\hline \multicolumn{1}{|c|}{\textbf{u\_p\_indx}} & \multicolumn{1}{|c|}{\textbf{name}} & \multicolumn{1}{|c|}{\textbf{allowed\_host}} \\ \hline \hline  \endfirsthead 
\caption{Content of table user\_privileges (continued)} \\ \hline \multicolumn{1}{|c|}{\textbf{u\_p\_indx}} & \multicolumn{1}{|c|}{\textbf{name}} & \multicolumn{1}{|c|}{\textbf{allowed\_host}} \\ \hline \hline \endhead \endfoot
1 & guest & all \\ \hline 
2 & basic & all \\ \hline 
3 & full & all \\ \hline 
4 & admin & all \\ \hline 
5 & config & all \\ \hline 
20 & TempControl & 192.168.1.14 \\ \hline 
25 & DAQ &  \\ \hline 
27 & TS & all \\ \hline 
17 & siren & all \\ \hline 
 \end{longtable}

%
% Structure: user_shift_status
%
 \begin{longtable}{|l|c|c|c|l|} 
 \caption{Structure of table user\_shift\_status} \label{tab:user_shift_status-structure} \\
 \hline \multicolumn{1}{|c|}{\textbf{Column}} & \multicolumn{1}{|c|}{\textbf{Type}} & \multicolumn{1}{|c|}{\textbf{Null}} & \multicolumn{1}{|c|}{\textbf{Default}} & \multicolumn{1}{|c|}{\textbf{Comments}} \\ \hline \hline
\endfirsthead
 \caption{Structure of table user\_shift\_status (continued)} \\ 
 \hline \multicolumn{1}{|c|}{\textbf{Column}} & \multicolumn{1}{|c|}{\textbf{Type}} & \multicolumn{1}{|c|}{\textbf{Null}} & \multicolumn{1}{|c|}{\textbf{Default}} & \multicolumn{1}{|c|}{\textbf{Comments}} \\ \hline \hline \endhead \endfoot 
\textbf{\textit{status}} & varchar(16) & No &  \\ \hline 
can\_manage & tinyint(1) & No & 0 \\ \hline 
 \end{longtable}

%
% Data: user_shift_status
%
 \begin{longtable}{|l|l|} 
 \hline \endhead \hline \endfoot \hline 
 \caption{Content of table user\_shift\_status} \label{tab:user_shift_status-data} \\\hline \multicolumn{1}{|c|}{\textbf{status}} & \multicolumn{1}{|c|}{\textbf{can\_manage}} \\ \hline \hline  \endfirsthead 
\caption{Content of table user\_shift\_status (continued)} \\ \hline \multicolumn{1}{|c|}{\textbf{status}} & \multicolumn{1}{|c|}{\textbf{can\_manage}} \\ \hline \hline \endhead \endfoot
Off & 0 \\ \hline 
Shift Leader & 1 \\ \hline 
Shift Manager & 1 \\ \hline 
On Shift & 0 \\ \hline 
System Manager & 1 \\ \hline 
 \end{longtable}




\end{document}
